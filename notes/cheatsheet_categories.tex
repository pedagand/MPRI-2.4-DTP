\documentclass[usenames,dvipsnames,rgb]{beamer}

\usepackage[utf8x]{inputenc}
\usepackage{amssymb}
\usepackage{ifthen}
\usepackage{xintexpr}
\usepackage{tikz-cd}
\usepackage{mdframed}

\usetheme{default}
\usefonttheme[stillsansserifmath,stillsansserifsmall]{serif}

\beamertemplatenavigationsymbolsempty

\definecolor{frameblue}{RGB}{11,116,149}
\definecolor{titleblue}{RGB}{11,116,149}
\definecolor{framegray}{RGB}{204,204,204}

\setbeamercolor{title}{fg=titleblue}
\setbeamerfont{title}{size=\Large}

\setbeamercolor{frametitle}{fg=frameblue}
\setbeamercolor{page number in head/foot}{fg=white}
\setbeamercolor{block title}{fg=black}
\setbeamerfont{block title}{size=\large,family=\sffamily}

\mdfdefinestyle{blockframe}{linecolor=framegray,
                            skipabove=5pt, 
                            skipbelow=5pt}

\setbeamertemplate{block begin}
{
  \begin{mdframed}[style=blockframe]
    {
      \usebeamerfont{block title}
      \usebeamercolor[fg]{block title}
      \insertblocktitle
      \vskip2pt
    }
}
\setbeamertemplate{block end}{\end{mdframed}}
\setbeamertemplate{footline}[frame number]
\setbeamertemplate{itemize items}[circle]
\setbeamercolor{itemize item}{fg=black}


\newcommand{\sidenote}[1]{\hfill #1}
\newcommand{\law}[1]{\textit{(#1)}}
\newcommand{\Forall}[1]{\forall #1.\:}

%% Categorical macros

\begin{document}

%%%%%%%%%%%%%%%%%%%%%%%%%%%%%%%%%%%%%%%%%%%%%%%%%%%%%%%%%%%%%%%%

\newcommand{\theCategory}[1]{\mathbb{#1}}
\newcommand{\aCategory}[1][0]{%
  \ensuremath{%
    \ifthenelse%
        {\equal{#1}{0}}%
        {\theCategory{C}}%
        {\ifthenelse%
          {\equal{#1}{1}}%
          {\theCategory{D}}%
          {\ifthenelse%
            {\equal{#1}{2}}%
            {\theCategory{E}}%
            {\theCategory{C}_{\xinttheiexpr #1 - 3 \relax}}}}}}

\newcommand{\CAT}{\theCategory{C}\theCategory{A}\theCategory{T}}
\newcommand{\SET}{\theCategory{S}\theCategory{E}\theCategory{T}}

\newcommand{\CatObj}[1]{\ensuremath{| #1 |}}
\newcommand{\CatMorph}[3][]{%
  \ensuremath{%
    \ifthenelse{\equal{#1}{}}%
               {#2 \to #3}%
               {#1(#2, #3)}}}
\newcommand{\CatId}[1][]{\mathsf{id}_{#1}}

\begin{frame}
\frametitle{Category%
  \sidenote{$(\aCategory, \aCategory[1], \aCategory[2], \aCategory[3], \aCategory[4], \ldots \in \CAT)$}}

\begin{itemize}
\item A class of objects \CatObj{\aCategory}
  \sidenote{$(C, D, E, \ldots \in \CatObj{\aCategory})$}
\item A class of morphisms \CatMorph[\aCategory]{C}{D} from object C to object D\\
  \sidenote{$(f, g, h, \ldots \in \CatMorph[\aCategory]{C}{D})$}\\
  \sidenote{$(f, g, h, \ldots \in \CatMorph{C}{D})$}
\end{itemize}
with:
\begin{itemize}
\item an identity morphism $\CatId[C] \in \CatMorph[\aCategory]{C}{C}$
\item a composition operation $- \mathop{\circ} - \in \CatMorph[\aCategory]{D}{E} \to \CatMorph[\aCategory]{C}{D} \to \CatMorph[\aCategory]{C}{E}$
\end{itemize}
such that:
\begin{itemize}
\item $\CatId[C] \circ f = f \circ \CatId[D] = f$ %
  \sidenote{\law{unity, left \& right}}\\
  for any $f \in \CatMorph[\aCategory]{C}{D}$
\item $(f \circ g) \circ h = f \circ (g \circ h)$ %
  \sidenote{\law{associativity}}\\
  for any $h \in \CatMorph[\aCategory]{C}{D}$, 
          $g \in \CatMorph[\aCategory]{D}{E}$ and 
          $f \in \CatMorph[\aCategory]{E}{F}$ 
\end{itemize}

\end{frame}

%%%%%%%%%%%%%%%%%%%%%%%%%%%%%%%%%%%%%%%%%%%%%%%%%%%%%%%%%%%%%%%%

\newcommand{\CatOp}[1]{{#1}^{\mathsf{op}}}

\begin{frame}
\frametitle{Opposite category}

The opposite category $\CatOp{\aCategory}$ consists of:
\begin{itemize}
\item Objects $\CatObj{\CatOp{\aCategory}} = \CatObj{\aCategory}$
\item Morphisms $\CatMorph[\CatOp{\aCategory}]{C}{D} = \CatMorph[\aCategory]{D}{C}$
\end{itemize}


\end{frame}
%%%%%%%%%%%%%%%%%%%%%%%%%%%%%%%%%%%%%%%%%%%%%%%%%%%%%%%%%%%%%%%%

\newcommand{\TermObj}{\top}
\newcommand{\TermMorph}{\mathsf{tt}}

\begin{frame}[fragile]
  \frametitle{Terminal object}

  \begin{itemize}
  \item An object $\TermObj \in \CatObj{\aCategory}$
  \end{itemize}
  %
  such that for every
  %
  \begin{itemize}
  \item object $X \in  \CatObj{\aCategory}$
  \end{itemize}
  %
  there exists a unique morphism $\TermMorph \in \CatMorph[\aCategory]{X}{\TermObj}$.

  \begin{center}
    \begin{tikzcd}[column sep=large, row sep=large]
      X
        \arrow[d, dashed, "\exists ! \TermMorph"]
      \\
      \TermObj
    \end{tikzcd}
  \end{center}


\end{frame}

%%%%%%%%%%%%%%%%%%%%%%%%%%%%%%%%%%%%%%%%%%%%%%%%%%%%%%%%%%%%%%%%

\newcommand{\ProdObj}[2]{\ensuremath{#1 \times #2}}
\newcommand{\ProdMorph}[2]{\ensuremath{[ #1 \mathop{,} #2 ]}}
\newcommand{\ProdFst}{\pi_1}
\newcommand{\ProdSnd}{\pi_2}

\begin{frame}[fragile]
  \frametitle{Cartesian product}

  Let $C, D \in \CatObj{\aCategory}$.

  \vfill

  \begin{itemize}
  \item An object $\ProdObj{C}{D} \in \CatObj{\aCategory}$
  \item A morphism
    $\ProdFst \in \CatMorph[\aCategory]{\ProdObj{C}{D}}{C}$
  \item A morphism
    $\ProdSnd \in \CatMorph[\aCategory]{\ProdObj{C}{D}}{D}$
  \end{itemize}
  %
  such that for every
  %
  \begin{itemize}
  \item $X \in \CatObj{\aCategory}$
  \item $f \in \CatMorph[\aCategory]{X}{C}$
  \item $g \in \CatMorph[\aCategory]{X}{D}$
  \end{itemize}
  %
  there exists a unique morphism $\ProdMorph{f}{g} \in \CatMorph[\aCategory]{X}{\ProdObj{C}{D}}$ verifying
  %
  \begin{center}
    \begin{tikzcd}[column sep=large, row sep=large]
        &
      X
        \arrow[ddl, bend right, "\forall f"{above left}]
        \arrow[ddr, bend left, "\forall g"]
        \arrow[d, dashed, "\exists ! \ProdMorph{f}{g}"]
        &
        \\
        &
      \ProdObj{C}{D}
        \arrow[dl, "\pi_1"]
        \arrow[dr, "\pi_2"{below left}]
        &
        \\
      C &             & D \\
    \end{tikzcd}
  \end{center}

  \vfill

\end{frame}

%%%%%%%%%%%%%%%%%%%%%%%%%%%%%%%%%%%%%%%%%%%%%%%%%%%%%%%%%%%%%%%%

\newcommand{\ExpObj}[2]{\ensuremath{{#1}^{#2}}}
\newcommand{\ExpMorph}[1]{\ensuremath{\lambda\; #1}}
\newcommand{\ExpApply}{\ensuremath{\mathsf{apply}}}

\begin{frame}[fragile]
  \frametitle{Exponential}

  Let $C \in \CatObj{\aCategory}$.
  Let $D \in \CatObj{\aCategory}$ with all Cartesian products.

  \vfill

  \begin{itemize}
  \item An object $\ExpObj{C}{D} \in \CatObj{\aCategory}$
  \item A morphism
    $\ExpApply \in \CatMorph[\aCategory]{\ProdObj{\ExpObj{C}{D}}{D}}{C}$
  \end{itemize}
  %
  such that for every
  %
  \begin{itemize}
  \item $X \in \CatObj{\aCategory}$
  \item $f \in \CatMorph[\aCategory]{\ProdObj{X}{D}}{C}$
  \end{itemize}
  %
  there exists a unique morphism $\ExpMorph{f} \in \CatMorph[\aCategory]{X}{\ExpObj{C}{D}}$ verifying
  %
  \begin{center}
    \begin{tikzcd}[column sep=large, row sep=large]
      X
        \arrow[d, dashed, "\exists ! \ExpMorph{f}"{left}]
        &
      \ProdObj{X}{D}
        \arrow[dr, "\forall f"{above right}]
        \arrow[d, dashed, "\ProdMorph{\ExpMorph{f}}{\CatId[D]}"{left}]
        &
        \\
      \ExpObj{C}{D}
        &
      \ProdObj{\ExpObj{C}{D}}{D}
        \arrow[r, "\ExpApply"{below}]
        &
      C
    \end{tikzcd}
  \end{center}

  \vfill

\end{frame}

%%%%%%%%%%%%%%%%%%%%%%%%%%%%%%%%%%%%%%%%%%%%%%%%%%%%%%%%%%%%%%%%

\begin{frame}[fragile]
  \frametitle{Cartesian-closed category}

  A category \aCategory{} with
  %
  \begin{itemize}
  \item A terminal object $\TermObj$
  \item A Cartesian product $\ProdObj{C}{D}$ for any $C, D \in \CatObj{\aCategory}$
  \item An exponential object $\ExpObj{C}{D}$ for any $C, D \in \CatObj{\aCategory}$
  \end{itemize}

\end{frame}

%%%%%%%%%%%%%%%%%%%%%%%%%%%%%%%%%%%%%%%%%%%%%%%%%%%%%%%%%%%%%%%%

\newcommand{\CatFunc}[2]{\ensuremath{#1 \Rightarrow #2}}

\newcommand{\theFunctor}[1]{\mathcal{#1}}
\newcommand{\aFunctor}[1][0]{%
  \ensuremath{%
    \ifthenelse%
        {\equal{#1}{0}}%
        {\theFunctor{F}}%
        {\ifthenelse%
          {\equal{#1}{1}}%
          {\theFunctor{G}}%
          {\theFunctor{F}_{\xinttheiexpr #1 - 2 \relax}}}}}

\newcommand{\FuncObj}[1]{\CatObj{#1}}
\newcommand{\FuncMorph}[1]{#1^{\to}}

\begin{frame}
\frametitle{Functor%
  \sidenote{$(\aFunctor, \aFunctor[1], \aFunctor[2], \aFunctor[3], \ldots \in \CatFunc{\aCategory}{\aCategory[1]})$}}

\begin{itemize}
\item An action on objects $\FuncObj{\aFunctor} \in \CatObj{\aCategory} \to \CatObj{\aCategory[1]}$
\item An action on morphisms $\FuncMorph{\aFunctor} \in \CatMorph[\aCategory]{C}{C'} \to \CatMorph[{\aCategory[1]}]{D}{D'}$
\end{itemize}
such that:
\begin{itemize}
\item $\FuncMorph{\aFunctor}\: \CatId[C] = \CatId[{\FuncObj{\aFunctor}\: C}]$
\sidenote{\law{identity preservation}}
\item $\FuncMorph{\aFunctor}\: (g \circ f) = (\FuncMorph{\aFunctor}\: g) \circ (\FuncMorph{\aFunctor}\: f)$
\sidenote{\law{composition preservation}}
\end{itemize}

\end{frame}

%%%%%%%%%%%%%%%%%%%%%%%%%%%%%%%%%%%%%%%%%%%%%%%%%%%%%%%%%%%%%%%%

\begin{frame}[fragile]
  \frametitle{Adjunction}

  Let $\aFunctor \in \CatFunc{\aCategory[1]}{\aCategory}$.

  \aFunctor{} is a left adjoint if for every
  %
  \begin{itemize}
  \item Object $C \in \CatObj{\aCategory}$
  \end{itemize}
  %
  there exists
  %
  \begin{itemize}
  \item An object $\FuncObj{\aFunctor[1]} C \in \CatObj{\aCategory[1]}$
  \item A morphism $\epsilon_C \in \CatMorph[\aCategory]{\FuncObj{\aFunctor}(\FuncObj{\aFunctor[1]}(C))}{C}$
  \end{itemize}
  %
  such that for every
  %
  \begin{itemize}
  \item object $D \in \CatObj{\aCategory[1]}$
  \item morphism $f \in \CatMorph[\aCategory]{\FuncObj{\aFunctor}\: D}{C}$
  \end{itemize}
  %
  there exists a unique morphism $g \in \CatMorph[{\aCategory[1]}]{D}{\FuncObj{\aFunctor[1]}\:C}$
  %
  with
  %
\begin{center}
\begin{tikzcd}
  D
    \arrow[d, dashed, "\exists ! g"{left}]
  &
  \FuncObj{\aFunctor}\: D
    \arrow[dr, "\forall f"]
    \arrow[d, dashed, "\FuncMorph{\aFunctor}\: g"{left}]
  &
  \\
  \FuncObj{\aFunctor[1]}(C)
  &
  \FuncObj{\aFunctor}(\FuncObj{\aFunctor[1]}(C))
    \arrow[r, "\epsilon_C"{below}]
  &
  C
\end{tikzcd}
\end{center}


\end{frame}

%%%%%%%%%%%%%%%%%%%%%%%%%%%%%%%%%%%%%%%%%%%%%%%%%%%%%%%%%%%%%%%%

\newcommand{\CatNatTrans}[2]{\ensuremath{#1 \stackrel{\cdot}{\Rightarrow} #2}}

\newcommand{\aNatTrans}[1][0]{%
  \ensuremath{%
    \ifthenelse%
        {\equal{#1}{0}}%
        {\varphi}%
        {\ifthenelse%
          {\equal{#1}{1}}%
          {\psi}%
          {\varphi_{\xinttheiexpr #1 - 2 \relax}}}}}

\begin{frame}[fragile]
\frametitle{Natural transformation%
  \sidenote{$(\aNatTrans, \aNatTrans[1], \aNatTrans[2], \aNatTrans[3], \ldots \in \CatNatTrans{\aFunctor}{\aFunctor[1]})$}}

Let $\aFunctor, \aFunctor[1] \in \CatFunc{\aCategory}{\aCategory[1]}$.

\vfill 

\begin{itemize}
\item A transformation %
  $\aNatTrans \in \Forall{C}%
       \CatMorph[{\aCategory[1]}]%
                {\FuncObj{\aFunctor}\: C}%
                {\FuncObj{\aFunctor[1]}\: C}$
\end{itemize}
such that for every
\begin{itemize}
\item $k \in \CatMorph[\aCategory]{C}{D}$
\end{itemize}
we have

\begin{center}
\begin{tikzcd}
  \FuncObj{\aFunctor}\: C 
    \arrow[r, "\aNatTrans_C"]
    \arrow[d, "\FuncMorph{\aFunctor}\: k"{left}] 
  &   \FuncObj{\aFunctor[1]}\: C
      \arrow[d, "\FuncMorph{\aFunctor[1]}\: k"] 
  \\
  \FuncObj{\aFunctor}\: D
    \arrow[r, "\aNatTrans_D"{below}]
  & \FuncObj{\aFunctor[1]}\: D
\end{tikzcd}
\end{center}

\vfill

\end{frame}


%%%%%%%%%%%%%%%%%%%%%%%%%%%%%%%%%%%%%%%%%%%%%%%%%%%%%%%%%%%%%%%%

\newcommand{\Adj}{\mathop{\dashv}}

\begin{frame}
  \frametitle{Adjunction, take II}

  Let $\aFunctor \in \CatFunc{\aCategory[1]}{\aCategory}$ and
  $\aFunctor[1] \in \CatFunc{\aCategory}{\aCategory[1]}$.

  \vfill

  \aFunctor{} is left adjoint to \aFunctor[1] ($\aFunctor \Adj
  \aFunctor[1]$) if there exists
  %
  \begin{itemize}
  \item a natural isomorphism $\aNatTrans \in \CatMorph[\aCategory]{\FuncObj{\aFunctor}\: D}{C} \stackrel{\sim}{\to}
                                              \CatMorph[{\aCategory[1]}]{D}{\FuncObj{\aFunctor[1]}\: C}$
  \end{itemize}

  \vfill

  Conversely, \aFunctor[1] is right adjoint to \aFunctor{}.


\end{frame}

%%%%%%%%%%%%%%%%%%%%%%%%%%%%%%%%%%%%%%%%%%%%%%%%%%%%%%%%%%%%%%%%

\newcommand{\aMonad}[1][]{{\theFunctor{M}}_{#1}}

\begin{frame}[fragile]
  \frametitle{Monad}

  \begin{itemize}
  \item A functor $\aMonad \in \CatFunc{\aCategory}{\aCategory}$
  \item A natural transformation $\eta \in \Forall{C} \CatMorph{C}{\FuncObj{\aMonad}\: C}$
  \item A natural transformation $\mu \in \Forall{C} \CatMorph{\FuncObj{\aMonad}(\FuncObj{\aMonad}\: C)}{\FuncObj{\aMonad}\: C}$
  \end{itemize}
  %
  such that
  %
  \begin{columns}
    \begin{column}{.5\textwidth}
      \begin{tikzcd}[column sep=large, row sep=large]
        \FuncObj{\aMonad}^3\: C
        \arrow[d, "\mu_{\FuncObj{\aMonad}\: C}"{left}]
        \arrow[r, "\FuncObj{\aMonad} \mu_{C}"]
        &
        \FuncObj{\aMonad}^2\: C
        \arrow[d, "\mu_{C}"]
        \\
        \FuncObj{\aMonad}^2\: C
        \arrow[r, "\mu_{C}"]
        &
        \FuncObj{\aMonad}\: C
      \end{tikzcd}
    \end{column}
    \begin{column}{.5\textwidth}
      \begin{tikzcd}[column sep=large, row sep=large]
        \FuncObj{\aMonad}\: C
          \arrow[d, "\eta_C"]
          \arrow[r, "\eta_{\FuncObj{\aMonad}\: C}"]
          \arrow[dr, equal]
        &
        \FuncObj{\aMonad}^2\: C
          \arrow[d, "\mu_C"]
        \\
        \FuncObj{\aMonad}^2\: C
          \arrow[r, "\mu_C"]
        &
        \FuncObj{\aMonad}\: C
      \end{tikzcd}
    \end{column}
  \end{columns}
\end{frame}


%%%%%%%%%%%%%%%%%%%%%%%%%%%%%%%%%%%%%%%%%%%%%%%%%%%%%%%%%%%%%%%%

\newcommand{\KleisliCat}[2]{\ensuremath{{#1}_{#2}}}

\begin{frame}
  \frametitle{Kleisli category}

  Let $\aMonad$ be a monad over $\aCategory$.

  \vfill

  The Kleisli category \KleisliCat{\aCategory}{\aMonad} consists of:
  \begin{itemize}
  \item Objects: $\FuncObj{\aMonad}\: C$ for every $C \in \CatObj{\aCategory}$
  \item Morphisms: $\CatMorph[\KleisliCat{\aCategory}{\aMonad}]{C}{C'} = \CatMorph[\aCategory]{C}{\FuncObj{\aMonad}\: C'}$
  %% \item Identity: $\CatId[C] = \eta \in \CatMorph[\KleisliCat{\aCategory}{\aMonad}]{C}{C}$
  %% \item Composition: $g \circ f = \mu \circ \FuncMorph{\aMonad}\: g \circ f$
  \end{itemize}

  \vfill

\end{frame}

%%%%%%%%%%%%%%%%%%%%%%%%%%%%%%%%%%%%%%%%%%%%%%%%%%%%%%%%%%%%%%%%

\newcommand{\Presheaf}[1]{\hat{#1}}

\begin{frame}
  \frametitle{Presheaf}

  A presheaf on a category \aCategory{} is a functor $\Presheaf{\aCategory} \in
  \CatFunc{\aCategory}{\SET}$.

  \bigskip
  
  In particular, it forms a functor category:
  \begin{itemize}
  \item Objects: presheaf functors
  \item Morphisms: natural transformations
  \end{itemize}

\end{frame}

%%%%%%%%%%%%%%%%%%%%%%%%%%%%%%%%%%%%%%%%%%%%%%%%%%%%%%%%%%%%%%%%


\begin{frame}
  \frametitle{Further reading}

  \begin{itemize}
  \item ``Conceptual Mathematics'', Schanuel \& Lawvere
  \item ``An introduction to Category Theory'', Simmons
  \item ``Categories for Types'', Crole
  \item ``Categories for the Working Mathematician'', Mac Lane
  \item ``Sheaves in Geometry and Logic'', Mac Lane \& Moerdijk
  \end{itemize}
\end{frame}

\end{document}
